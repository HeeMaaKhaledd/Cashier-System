% !TEX root = /Users/saxer/Documents/uppsalaDV/pkd/project/Cashier-System
% !TEX encoding = UTF-8 Unicode
%sets page
\documentclass[11pt]{article}
\usepackage[a4paper,top=3cm,bottom=2cm,left=3cm,right=3cm,marginparwidth=1.75cm]{geometry}
\usepackage[T1]{fontenc}
\usepackage[utf8]{inputenc}
\usepackage[swedish]{babel}



\begin{document}
\title{PKD project - Diary}
\author{Jesper Saxer, Sebastian Lådhö , Grim Moström}
\date{}
\maketitle
\section*{Wednesday, 15:e February}
We begin the day with a meeting where we brainstorm about what kind of project we should persue. These were the most attractive projects we came up with:
\begin{itemize}
  \item A graphical solution for displaying a binary tree in haskell but we couldn't find any fast solution for this within the timeframe.
  \item A QR-generator which encrypts a barcode or similar to a QR code for scanning including a decrypter, possible to use creating tickets for example
  \item A digitalized shop to upload items like, milk candy and so on. You should also be able to create an account with admin user etc and have a wallet connected to, where the money decreases as you buy things. Due to sebastian have a Ean scanner we thought this would be a suitable project which is very easy scale down or to do more advanced.
\end{itemize}
We choose the digital shop project as it is something we think is applicable to a real life system and something we could continue develop after the course. We believe this project will allow us to use what we learned throughout the course as well as give us a deeper understanding of some of the more advance thing we have encountered.\\\\
To divide the project among us we are going to use GitHub, which is a great opportunity for us to develop our knowledge about GitHub and become more familiar with the program and the functions it can offer. GitHub is allowing everyone to work on the project from different computers and proposing different solutions which we then discuss and decide which one to implement. In addition to using GitHub this we have agreed to split all the significant files to three parts so everyone can contribute to all files.\\\\
We have now emailed our supervisor Lei you about our idea and began to structure a framework for different functions and data types to build the shop.\\\\
-we are now waiting for decisions from Lei You whether our project is suitable or not.\\
Grim kommer inte att närvara under handledningstillfället då han befinner sig på annan ort, men kommer att bli briefad av gruppen efter mötet och bli delegerad arbetsuppgifter.\\\\
\section*{Thursday, February 16}
Today we created a skelleton, this way it will be easier to devide the work between us all. This will also be helpful when we start working from Git to see what kind of work each and everyone does.

\section*{Friday, February 17}
Everything is now up on Github and our progress forward has begun!
\section*{Sunday, February 19}
Finished the file User.hs
\section*{Monday, February 20}
Grim has returned from Umeå and everyone is currently coding on the project trying to work out what we should do next, and who should do what, also uploaded diary.tex to git. Added 8 testcases to Item.hs
\section*{Tuesday, February 21 , 4 hours }
Sebastian finished structuring up Interface.hs and writing down responsibilities for each member in the group.
Together we all sat down and finished writing database.hs however we are trying to figure how we was suppoesed to write a more efficient database structure... HashMap? Red-Black Tree?
\section*{Wednesday, February 22, 3 hours }
Today our main focus is to start working on Interface. We started delegate workload arount the group for the hours 10:00 - 13:15 \\
Grim got to handle addItem,DeleteItem,createUser,deleteUser \\
Sebastian started working on documentation, the functions addToCart,removeFromCart in Interface \\
Jesper got to handle addToStock,replaceStock,removeFromStock \\
\section*{Thursday, February 23}

\end{document}
