% !TEX root = /Users/saxer/Documents/uppsalaDV/pkd/project/Cashier-System
% !TEX encoding = UTF-8 Unicode
%sets page
\documentclass[11pt]{article}
\usepackage[a4paper,top=3cm,bottom=2cm,left=3cm,right=3cm,marginparwidth=1.75cm]{geometry}
\usepackage[T1]{fontenc}
\usepackage[utf8]{inputenc}
\usepackage[swedish]{babel}



\begin{document}
\title{PKD project - Diary}
\author{Jesper Saxer, Sebastian Lådhö , Grim Moström}
\date{}
\maketitle
\section*{Wednesday, 15:e February}
Dagen inleds med uppstart av möte: Sebastian, Jesper och Grim Har tillsammans kommit med förslag på olika projekt som kan uppstartas och som bör genomgås.
Några av de förslag som kommit upp:
En grafisk framställning av de Binära träd som skapats under kursens gång, då vi uppfattat ett behov av att praktiskt illustrera denna process, och att det inte verkar finnas tillgängligt.\\

En QR-tillverkare, som krypterar en streck-kod till et QRmönster. Med denna typ av QR mönster kan man scanna in nya produkter på marknaden, e-biljetter osvosv. Mycket trevligt…\\

En digital butik/företagsverktyg att lägga upp varor och priser etc. För att på ett mer stabilt sätt kunna lagerföra och hålla försäljning av produkter utan att manuellt/analogt bokföra varorna. Sebastian har en IR-streckkodsscanner hemma, vilket gjorde det intressant då det ger bättre återkoppling till verkligheten och att externa instrument används i relation till datavetenskapen. Ett mycket skalbart projekt vilket ger det tillgänglighet att anpassa till projektets tidsram och storlek.\\

En Krypterad chatt som tillåter avsändare och motagare att kryptera sina meddelanden i överföring mellan datorer. På så sätt skulle man kunna överföra säker information på ett komprimerat och krypterat sätt.\\

Som resonemanget har gått vill vi alla kunna koppla våra skapelser till det “verkliga livet” och i någon form skapa näringslivskoppling. Vi har valt att skapa en digital butik/lagerförare. Med denna idé har vi chansen att prestera till det bästa av våra förmågor givet den tidsram som erbjudits, samt kunna presentera succesivt klara lösningar och funktioner inuti programmet.\\

I uppstartsskedet har vi resonerat över det optimala sättet att dela processer, ideer och kommunikation på distans och praktiskt utförande. Vi har kommit fram till att vi bör sätta oss in i bättre hur Github fungerar då en i gruppen har en privat server som tillåter oss att dela all information privat inom gruppen. I addition till detta har vi kommit överens om att vi har för ambition att dela arbetsbördan lika mellan alla parter och kommunicera de tillfälliga begränsningar som kan råda och begränsa processen.\\

Vi har nu mailat vår handledare Lei You om vår idé och börjat med en mall för de olika funktioner och datatyper som måste finnas i ramverket för att kunna tillverka systemet.
-Besked återstår vad handledaren säger angående idén\\

Grim kommer inte att närvara under handledningstillfället då han befinner sig på annan ort, men kommer att bli briefad av gruppen efter mötet och bli delegerad arbetsuppgifter.\\\\
\section*{Thursday, February 16}
Today we created a skelleton, this way it will be easier to devide the work between us all. This will also be helpful when we start working from Git to see what kind of work each and everyone does.

\section*{Friday, February 17}
Everything is now up on Github and our progress forward has begun!
\section*{Sunday, February 19}
Finished the file User.hs
\section*{Monday, February 20}
Grim has returned from Umeå and everyone is currently coding on the project trying to work out what we should do next, and who should do what, also uploaded diary.tex to git. Added 8 testcases to Item.hs


\end{document}
