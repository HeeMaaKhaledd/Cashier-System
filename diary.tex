% !TEX root = /Users/saxer/Documents/uppsalaDV/pkd/project/Cashier-System/diary.tex
% !TEX encoding = UTF-8 Unicode
%sets page
\documentclass[11pt]{article}
\usepackage[a4paper,top=3cm,bottom=2cm,left=3cm,right=3cm,marginparwidth=1.75cm]{geometry}
\usepackage[T1]{fontenc}
\usepackage[utf8]{inputenc}



\begin{document}
\title{PKD project - Diary}
\author{Jesper Saxer, Sebastian Lådhö , Grim Moström}
\date{}
\maketitle
\section*{Wednesday, 15:e February, 3 hours}
We begin the day with a meeting where we brainstorm about what kind of project we should persue. These were the most attractive projects we came up with:
\begin{itemize}
  \item A graphical solution for displaying a binary tree in haskell but we couldn't find any fast solution for this within the timeframe.
  \item A QR-generator which encrypts a barcode or similar to a QR code for scanning including a decrypter, possible to use creating tickets for example
  \item A digitalized shop to upload items like, milk candy and so on. You should also be able to create an account with admin user etc and have a wallet connected to, where the money decreases as you buy things. Due to sebastian have a Ean scanner we thought this would be a suitable project which is very easy scale down or to do more advanced.
\end{itemize}
We chose the digital shop project as it is something we think is applicable to a real life system and something we could continue develop after the course. We believe this project will allow us to use what we learned throughout the course as well as give us a deeper understanding of some of the more advance thing we have encountered.\\\\
To divide the project among us we are going to use GitHub, which is a great opportunity for us to develop our knowledge about GitHub and become more familiar with the program and the functions it can offer. GitHub is allowing everyone to work on the project from different computers and proposing different solutions which we then discuss and decide which one to implement. In addition to using GitHub this we have agreed to split all the significant files to three parts so everyone can contribute to all files.\\\\
We have now emailed our supervisor Lei You about our idea and began to structure a framework for different functions and data types to build the shop.\\\\
-we are now waiting for decisions from Lei You whether our project is suitable or not.\\
\section*{Thursday, February 16, 4 hours}
We sat down and configured google docs for the project! \\
Today a email from Lei You came letting us know we got the good to go on our project!
we decided to start working directly with the project and wrote a big part of a great skeleton we could work with.
To create a skeleton for a shop we sat down for a couple hours just brainstorming about functions that will be needed in the system.
\section*{Friday, February 17, 1 hour}
Today Grim had to go to Umeå so we decided to not work to much on the project today.
But we gave every one home assignment to think more about functions need in the system
\section*{Sunday, February 19, 2 hours}
Jesper and Sebastian sat down and wrote a mockup of user.hs.
\section*{Monday, February 20, 4 hours}
Grim returned from umeå and we met up in foobar and started working on creating a flowchart for the system itself. We created flowchart.png which we then started working towards creating in code. \\
This ment creating the files Database,Item,User,Cart,Interface,menu... and importing them the correct way.
\section*{Tuesday, February 21 , 4 hours }
Sebastian finished structuring up Interface.hs and writing down responsibilities for each member in the group.
Together we all sat down and finished writing database.hs however we are trying to figure how we was supposed to write a more efficient database structure... HashMap? Red-Black Tree?
\section*{Wednesday, February 22, 3 hours }
Today our main focus is to start working on Interface. We started delegate workload around the group for the hours 10:00 - 13:15 \\
Grim got to handle addItem,DeleteItem,createUser,deleteUser \\
Sebastian started working on documentation, the functions addToCart,removeFromCart in Interface \\
Jesper got to handle addToStock,replaceStock,removeFromStock \\
\section*{Thursday, February 23, 8 hours}
Today we had a rough day! As both Jesper and Grim managed to finish their part of interface, Sebastian got stuck on the buy function.
This forced us all to sit down and talk over a new solution, we came to the conclusion that a new data structure was needed! So we implemented the new data structure. \\
However with this new data structure implemented we came to the realization that our whole Interface.hs had to be rewritten...\\
So we delegated the workload on rewriting Interface and sat in foobar until it was finished!
\section*{Friday, February 24, 9 hours}
Grim started writing a large part of the Project report! But also wrote some coding convention skeletons!
Jesper started working on a skeleton for the project report in LaTeX, because we can't turn in a google doc... that would feel wrong.
Sebastian dedicated his time on writing coding convention and also looking into creating a ui!
\section*{Saturday, February 25, 7 hours}
Today we all sat down and talked over what is need to be done before deadline!
We decided to focus on the main functions and get coding convention done! we all sat down and started writing coding convention!
After a while we concluded that a break from coding convention was needed therefore we started writing menu.hs
\section*{Sunday, February 26, 5 hours}
Both Jesper and Grim wrote testcases for all the basic functions in Database.hs, User.hs ,Cart.hs , Item.hs
Sebastian checked over interface.hs and found that it returned a lot of errors because of user inputs. Therefore he wrote Check.hs and added if cases for the entire menu.hs
\section*{Monday, February 27, 7 hours}
Today we all worked on the Project Report and finished up the style of menu.hs. We also looked over the code in every file and fixed all the things that needed fixing.
\end{document}
